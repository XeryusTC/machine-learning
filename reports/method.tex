To learn how to play Hare and Hounds we will use
Q-learning\cite{watkins1992q}.  Each player will have his own different
Q-table to store all values. This means that the actions of the hare are
separate from those of the three hounds combined. We did not opt for using
a separate Q-table for each hound, because the hounds are supposed to be
working together as they are the pieces of a single player. Furthermore,
the states are indifferent to which hound is in which spot. Our state space
consists of all different positions of the hounds and the hare, where all
permutations of hounds are considered the same state.

Each player also has their own set of actions $A$. At each turn, the possible
actions are generated for the player. To select the most appropriate action we
use  the softmax function with a temperature. This means that during training
we can move from exploration to exploitation. When the temperature is infinite
then  pure exploration is used while it would ensure pure exploitation when its
value is very close to zero. The action selection function then looks as
follows
\[ P(a|s) = \frac{\exp[Q(s,a)/T]}{\sum_{b \in A} \exp[Q(s, b)/T]} \]
where $P(a|s)$ is the probability that action $a$ is selected given the
current state $s$, $A$ is the set of available actions in a state, $Q(s,a)$ is
the value from the Q-table for a state-action pair $(s,a)$ and $T$ is the
temperature parameter. At the start of training, the temperature is equal to
the number of games that are played. During training we linearly decrease the
temperature by a single degree after every game, the final temperature in the
last game is one degree.

We train both players by letting them play against each other while they have
the same parameters. The parameters are the learning rate $\eta$ and the
discount factor $\gamma$. The discount factor is constant during training but
$\eta$ is gradually lowered to improve conversion to the optimal
Q-values\cite{alpaydin}. We lower the value of $\eta$ by a constant amount so
that it equal to zero in the last round of training. We train using the
Q-learning algorithm that is shown in Algorithm \autoref{alg:Qlearning}.

A single round of training consists of both players playing the game until one
of them wins or they both have played 50 turns but neither has won yet. We call
this a tie and both players receive a small negative reward. Each player uses
the current $T, \eta, \gamma$ parameters. The Q-table gets updated after each
the opponents so that we do not have to remember the moves that we have but we
do only update the states that have been visited since the outcome of an action
is non-deterministic. Updating the Q-table after the opponents turn also
ensures that a player learns how the opponent reacts to a move. After the
training game is finished then the values for $T$ and $\eta$ are updated and
we start the next training game.

We train the players with different values of $\eta$ and $\gamma$ for a varying
number of games. We do not know how many games are needed to for training to
converge. This also means that we can take a look at how training progresses
over the training period. After training we test both of the players by letting
them play against each other for 1000 games. Because we know that the hounds
player has a slight advantage we expect this player to win the majority of the
games.

\begin{algorithm}
\caption{Q-learning algorithm\cite{alpaydin}}
\label{alg:Qlearning}
\begin{algorithmic}[1]
\State{Initialize all $Q(s, a)$ to 0}
\ForAll{episodes}
	\State{Initialize $s$}
	\Repeat
		\State{Choose $a$ using policy derived from $Q$ with softmax}
		\State{Take action $a$, observe $r$ and $s^\prime$}
		\State{$Q(s,a)\gets Q(s,a) + \eta(r + \gamma\max_{a^\prime}
			Q(s^\prime,a^\prime) - Q(s,a))$}
		\State{$s\gets s^\prime$}
	\Until{$s$ is terminal}
        \State{Lower $\eta$ and $T$}
\EndFor
\end{algorithmic}
\end{algorithm}

\subsection{Experimental design}
To test the implementation, we trained the system for multiple sets of
parameters and used the trained Q values to have the players play multiple
games against each other. We then kept track of how often each player won
and used this as our measure of performance. 

The full set of configurations can be found in \autoref{tbl:conf}. These
configurations were chosen to test a wide range of values of the
parameters, to get a good insight into the exact effects they would have on
the algorithm and the end result. 

\begin{table}
    \centering
    \caption{Configurations used during training. Each value of a parameter
        was matched with all other parameters.}
    \label{tbl:conf}
    \begin{tabular}{l|rrrrrrrr}
        \hline
        parameter & \\
        \hline
        runs & 100 & 1000 & 10000 & 100000 & 5000000 \\
        discount factor $\gamma$ & 0.5 & 0.6 & 0.7 & 0.8 & 0.9 & 0.95 &
        0.99 \\
        learning rate $\eta$ & 0.01 & 0.05 & 0.1 & 0.15 & 0.2 & 0.25 & 0.3
        & 0.5 \\
        \hline
    \end{tabular}
\end{table}

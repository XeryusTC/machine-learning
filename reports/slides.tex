\documentclass{beamer}

\usetheme{Dresden}
\usecolortheme{beaver}

\title{Learning to play Hare and Hounds using Q learning}
\author{Xeryus Stokkel \and Siegrid Lenting \and Ren\'e Mellema}
\date{20 January 2016}

\begin{document}

\begin{frame}
    \maketitle
\end{frame}

% Introduction
%% Explanation Hare and Hounds (Siegrid)
%% Expectations (Rene)
%% Differences in implementation (Xeryus)

% Method  (Xeryus)
%% Q learning
%% Experiments
\section{Method}
\begin{frame}
	\frametitle{Method}
	\begin{itemize}
		\item Q-learning
		\item Use annealing
	\end{itemize}
	\[ \hat{Q}(s_t, a_t) = \hat{Q}(s_t, a_t) + \eta \left(r_{t+1} + \gamma \max_{a_{t+1}} Q(s_{t+1}, a_{t+1}) - Q(s_t, a_t)\right) \]
\end{frame}

\begin{frame}
	\frametitle{Experimental setup}
	\begin{itemize}
		\item Train with varying $\gamma, \eta$ to build models
		\item Compare performance of different models
	\end{itemize}
\end{frame}

% Results (Rene)
%% Statistics for parameters
%% Tie in with background

% Discussion (Siegrid)
%% Would longer training help?
%% How can we improve the system?

\end{document}
